\chapter{CO\textsubscript{2} RELEASED BY CARBONATE SEDIMENT PRODUCTION IN SOME COASTAL AREAS MAY OFFSET THE BENEFITS OF SEAGRASS "BLUE CARBON" STORAGE
}		\label{another chapter}

\section{Abstract}

Seagrass ecosystems have been identified as long-term carbon sinks whose conservation could serve as a tool to mitigate carbon emissions. Seagrasses alter landscapes in a way that stimulates carbon biosequestration, but discussions of their role in atmospheric CO\textsubscript{2} mitigation disregard the co-occurring inorganic carbon cycle, whose antagonist effect on CO\textsubscript{2} sequestration can buffer and potentially outweigh the effects of C\textsubscript{org} production on net carbon exchange with the atmosphere. This study examines the extent of both organic carbon (C\textsubscript{org}) and inorganic carbon (C\textsubscript{inorg}) stocks as proxies for long-term production and calcification in the poorly studied seagrass meadows of southeastern Brazil and compares values to Florida Bay (USA), a well-studied system known for both high autotrophy and calcification, representing extremes of CaCO\textsubscript{3} soil content. Seagrass soils in SE Brazil contain an average of 67.6 $\pm$ 14.7 Mg C\textsubscript{org} ha\textsuperscript{-1} in the top 1m, compared to an average of 175.0 $\pm$ 20.4 Mg C\textsubscript{org} ha\textsuperscript{-1} for their counterparts in Florida Bay. C\textsubscript{inorg} as CaCO\textsubscript{3} in SE Brazil averaged 141.5 $\pm$ 60.0 Mg C\textsubscript{inorg} ha\textsuperscript{-1} in the top meter of soil while the warmer, calcification­-promoting waters of Florida Bay had higher soil C\textsubscript{inorg} areal stock, averaging 754.6 $\pm$ 26.7 Mg C\textsubscript{inorg} ha\textsuperscript{-1}. When the CO\textsubscript{2} evasion related to CaCO\textsubscript{3} production is considered, seagrass ecosystems with high CaCO\textsubscript{3} content may have CO\textsubscript{2} sequestered via C\textsubscript{org} accumulation negated by CO\textsubscript{2} produced by calcification. These findings prompt the reconsideration of carbon inventory methods and encourage regionally- and community- specific assessments of CO\textsubscript{2} sequestration abilities of seagrass ecosystems.


\section{Introduction}
Concerns of increasing greenhouse gas emissions and potential mitigation strategies have driven a renewed interest in carbon sequestration abilities in natural ecosyst.....



\section{Materials and Methods}

Seagrass characteristics and underlying soi....

\bigskip
\noindent Data Analysis
\medskip


Stocks of inorganic and organic carbon were measured using methods reported ....

\bigskip
\hspace*{\fill}(1)	C\textsubscript{org stored} -  Ψ × C\textsubscript{inorg stored} = CO\textsubscript{2 net sequestered} \hspace*{\fill}
\bigskip

\noindent Where C\textsubscript{org} stored is the C\textsubscript{org} density in mol mL soil\textsuperscript{-1}, Ψ is the gas exchange:reaction ratio of CO\textsubscript{2} and CaCO\textsubscript{3} proposed by \citet{Smith:2013fv}, C\textsubscript{inorg} stored is the C\textsubscript{inorg} density in mol mL soil\textsuperscript{-1}, and CO\textsubscript{2} net sequestered is the moles of CO\textsubscript{2} sequestered in mL of soil.

\bigskip
\hspace*{\fill}(2)	Ca\textsuperscript{2+} + (1 + Ψ) HCO\textsubscript{3}\textsuperscript{-} + (1 - Ψ) OH\textsuperscript{-} = CaCO\textsubscript{3} + ΨCO\textsubscript{2} + H\textsubscript{2}O \hspace*{\fill}
\bigskip

\noindent For the shallow, coa...



\section{Results}

The seagrass species found in the sampling regions of SE Brazil was...

\section{Discussion}

Seagrasses meadows are typically autotrophic ecosystems whose positive net ecosystem production acts as a sink for CO\textsubscript{2} while producing and storing C\textsubscript{org} .......

\bigskip
\noindent Buffering capacity of the C\textsubscript{org}-C\textsubscript{inorg} reaction couplet
\medskip

Primary producers and calcifying organisms have previously been considered mutually beneficial. ...

\bigskip
\noindent Long-term balance of net C\textsubscript{org} and C\textsubscript{inorg} production
\medskip


The integrated effects of net calcification (calcification and CaCO\textsubscript{3} dissolution) a....

\bigskip
\noindent Conclusion
\medskip

Seagrasses are highly valued for nutrient processing and ...




\section{Acknowledgements}

This chapter was published in \textit{Limnology and Oceanography} with coauthors Joel Creed, Mariana Aguiar, and James Fourqurean. This research was funded by FAPERJ grant E-26/200.027/2015 “Doutorado-sanduíche Reverso” from the Foundation for Support of Research in the State of Rio de Janeiro and FIU LACC’s Tinker Field Research Grant. Joel Creed acknowledges financial support from FAPERJ E-26/201.286/2014 and Conselho Nacional de Desenvolvimento Científico e Tecnológico 307117/2014-6. This material was developed in collaboration with the Florida Coastal Everglades Long-Term Ecological Research program under National Science Foundation Grant No. DEB-1237517. Eugenia Zandonà and Marcelo Weksler provided logistical support in Brazil, while conversations with Christian Lopes and Tim Moulton helped inspire and develop the manuscript. David Barahona helped prepare and process samples for nutrient analysis. We are grateful to two anonymous reviewers who offered support and valuable comments. This is contribution 41 of the Marine Education and Research Center at FIU.
