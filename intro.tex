\chapter{INTRODUCTION}		\label{another chapter}

Current atmospheric concentrations of CO\textsubscript{2} are the highest in the last 800,000 years \citep{Luthi:2008wl}, and there is consensus that recent increases are due to human activity \citep{Solomon:2007wl}. Roughly 50 \% of the anthropogenic CO\textsubscript{2} released in recent times has been absorbed by the biosphere and the ocean (Le Quéré et al. 2009), leaving the remaining fraction in the atmosphere to increase the greenhouse effect and alter the climate \citep{Solomon:2007wl}. Though the ocean acts as a buffer against rising atmospheric CO\textsubscript{2}, the consequence is a decrease in oceanic water pH deemed “ocean acidification” \citep{Raven:2005wq, Orr:2005hn} that negatively affects a wide range of organisms \citep{Kroeker:2013if}. Coordinated global efforts are being made to mitigate CO\textsubscript{2} emissions, which have led to a renewed interest in evaluating the carbon storage and flux in the biosphere \citep{Canadell:2008ch, Mcleod:2011gs}. Terrestrial ecosystems, primarily forests, are responsible for absorbing an estimated 30 \% of fossil fuel-related CO\textsubscript{2} emissions through photosynthesis and net growth \citep{Canadell:2007vw}, and currently store the equivalent of twice the atmospheric carbon in their biomass \citep{Canadell:2008ch}. The alteration of terrestrial ecosystems through land use and land-cover change, accounts approximately 12.5 \% of total CO\textsubscript{2} emissions released into the atmosphere \citep{Houghton:2012ch}. The reduction of ecosystem degradation and destruction via land use change is an important component of global strategies to curb atmospheric CO\textsubscript{2} emissions, thus obligatory and voluntary carbon credit markets have been developed to add economic incentive to forest conservation \citep{Rizvi:2015ug}.......


\renewcommand{\thesection}{}
\renewcommand{\thesubsection}{\arabic{section}.\arabic{subsection}}
\makeatletter
\def\@seccntformat#1{\csname #1ignore\expandafter\endcsname\csname the#1\endcsname\quad}
\let\sectionignore\@gobbletwo
\let\latex@numberline\numberline
\def\numberline#1{\if\relax#1\relax\else\latex@numberline{#1}\fi}
\makeatother

\section{\normalfont{Work Cited}}
\begingroup
\setlength{\bibsep}{10pt}
\linespread{1}\selectfont

\renewcommand{\section}[2]{}%
\bibliographystyle{apalike}
\bibliography{wholebib_intro}
\endgroup
